\chapter{Algoritmul si codul numeric}

Algoritmul numeric const\u a \^{\i}n rezolvarea, in Mathematica a ecua\c tiei Schr\"odinger radial\u a (\ref{srad})  cu condi\c tii la limit\u a
\begin{align}
  \chi(r)\mathop{\sim}_{r\rightarrow 0}r^{l+1},\quad 
  \chi'(r)\mathop{\sim}_{r\rightarrow 0}(l+1)r^{l}
\end{align}
\c si calcul derivatei logaritmice la o distan\c t\u a destul de mare (dincolo de raza de ac\c tiune a poten\c tialului). Cu expresia derivatei logaritmice rezolv\u am sistemul (\ref{eq1}-\ref{eq2}) pentru a determina $\tan\delta_l$. Pentru $l=0$ \c si in limita de energii mici determin\u am lungimea de \^{\i}mpr\u a\c stiere.

Codurile Mathematica sunt listate \^{\i}n continuare.
%%% Local Variables:
%%% mode: latex
%%% TeX-master: "../RE"
%%% End:
