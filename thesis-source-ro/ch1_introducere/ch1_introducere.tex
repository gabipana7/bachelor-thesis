\chapter{Introducere}

Ciocnirile dintre particule la energii joase \c{s}i ultra-joase ocup\u{a} un loc strategic la interesec\c{t}ia mai multor subiecte de cercetare \^{i}n chimia fizic\u{a}, \^{i}n fizica atomic\u{a}, molecular\u{a}, optic\u{a} \c{s}i  fizica st\u{a}rii condensate. Natura acestor ciocniri are un impact esen\c{t}ial \^{i}n studiul proceselor ce au loc la scar\u{a} atomic\u{a}, dintre care enumer\u{a}m: manipularea optic\u{a} a proceselor inelastice si reactive, m\u{a}surarea cu precizie a parametrilor atomici \c{s}i moleculari, coeren\c{t}ele materie-und\u{a} \c{s}i statistica condensa\c{t}ilor cuantici forma\c{t}i din atomi ce interac\c{t}ioneaz\u{a} slab. Aceste c\^{a}teva exemple sunt suficiente pentru a justifica interesul  \^{i}n studiul ciocnirilor la energii joase.\\


\^{I}n procesul de r\u{a}cire a gazelor p\^{a}n\u{a} la temperaturi foarte joase abilitatea de a controla interac\c{t}iunile dintre atomii gazului respectiv este esen\c{t}ial\u{a}. Cu c\^{a}t densitatea scade, cu atat interac\c{t}iile de dou\u{a} corpuri (doi atomi) devin predominante \c{s}i ele pot fi descrise \^{i}n termeni de ciocniri. \^{I}n lucrarea de fa\c{t}\u{a} ne vom ocupa de ciocnirile elastice dintre atomi. Acestea sunt esen\c{t}iale pentru atingerea echilibrului termic.\\

La temperaturi foarte joase, descrierile ce folosesc  teoria c\^{a}mpului mediu  ale gazelor cuantice degenerate depind de un num\u{a}r foarte mic de parametri de ciocnire. De exemplu, forma \c{s}i dinamica condensatelor Bose-Einstein depind doar de lungimea de \^{i}mpr\u{a}\c{s}tiere, definit\u a \^{\i}n cadrul teoriei ciocnirilor.\\

Trebuie men\c{t}ionat faptul c\u{a} este posibil de controlat interac\c{t}ia atom-atom \c{s}i prin rezonan\c{t}e Feshbach, pe care le putem observa prin calculul defazajelor.\\

\^{I}n aceas\u{a} lucrare vom calcula defazajele, apoi lungimea de \^{i}mpr\u{a}\c{s}tiere \^{i}n limita de energie zero pentru c\^{a}teva poten\c{t}iale simple, apoi pentru un poten\c{t}ial de interac\c{t}ie aproximat penru ciocnirile atomilor de Cesiu. Acesta con\c{t}ine dou\u{a} p\u{a}r\c{t}i: o component\u{a} de raz\u{a} scurt\u{a} ce dispare la o anumit\u{a} lungime \c{s}i o component\u{a} ce const\u{a} \^{i}ntr-o superpozi\c{t}ie de termeni Van der Waals.\\

%Realiz\u{a}m calculul \c{t}in\^{a}nd cont c\u{a} exist\u{a} solu\c{t}ii analitice pentru ecua\c{t}iile Schr\"{o}dinger radiale la energie nul\u{a} pentru poten\c{t}iale simple, folosite \^{i}n fizica atomic\u{a}, deci contribu\c{t}iile lungimilor de \^{i}mpr\u{a}\c{s}tiere pot fi g\u{a}site exact. A\c{s}adar putem g\u{a}si acest parametru \c{s}i pentru un poten\c{t}ial mai complex, apropiat de realitate.


Structura tezei este urm\u atoarea: \^{\i}n capitolul 2 prezent\u am elemente de teoria ciocnirilor \^{\i}n mecanica clasic\u a \c si \^{\i}n fizica cuantic\u a. Discut\u am solu\c tia de \^{\i}mpr\u a\c stiere, modul de ob\c tinere a acesteia  cu ajutorul fun\c tiei Green, aproxima\c tia Born, descompunerea solu\c tiei  de \^{\i}mpr\u a\c stiere \^{\i}n unde par\c tiale \c si \^{\i}n final definirea defazajelor \c si a lungimii de \^{\i}mpr\u a\c stiere. In capitolul 3 prezent\u a exemple numerice pentru c\^ateva poten\c tiale model;  metoda numeric\u a se bazeaz\u a pe un cod scris in Mathematica, ce este prezentata \^{\i}n anexa B. Anexa A con\c tine propriet\u a\c ti ale func\c tiilor Bessel sferice.




%%% Local Variables:
%%% mode: latex
%%% TeX-master: "../RE"
%%% End:
